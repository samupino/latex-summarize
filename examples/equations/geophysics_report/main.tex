\documentclass{georeport}

% An example of defining macros
\newcommand{\rs}[1]{\mathstrut\mbox{\scriptsize\rm #1}}
\newcommand{\rr}[1]{\mbox{\rm #1}}

\begin{document}

\title{An example \emph{Geophysics} report, \\ with a two-line title}

\renewcommand{\thefootnote}{\fnsymbol{footnote}} 

\ms{GEO-Example} % manuscript number

\address{
\footnotemark[1]BP UTG, \\
200 Westlake Park Blvd, \\
Houston, TX, 77079 \\
\footnotemark[2]Bureau of Economic Geology, \\
John A. and Katherine G. Jackson School of Geosciences \\
The University of Texas at Austin \\
University Station, Box X \\
Austin, TX 78713-8924}
\author{Joe Dellinger\footnotemark[1] and Sergey Fomel\footnotemark[2]}

\footer{Example}
\lefthead{Dellinger \& Fomel}
\righthead{\emph{Geophysics} example}

\maketitle

\begin{abstract}
  This is an example of using \textsf{georeport.cls} for writing
  \emph{Geophysics} reports.
\end{abstract}

\section{Introduction}

This is an introduction. \LaTeX\ is a \new{powerful} document
typesetting system \cite[]{lamport}. An excellent reference is
\cite[]{kopka}. The new \textsf{geophysics.cls} class complies with
the \LaTeX2e\ standard. \old{We had something else written here but
  decided to get rid of it}.
\section*{Theory}

This is another section.

\subsection{Equations}

Section headings should be capitalized. Subsection headings should
only have the first letter of the first word capitalized.

Here are examples of equations involving vectors and tensors:
\begin{equation}
\tensor{R} = 
\pmatrix{R_{\rs{XX}} & R_{\rs{YX}} \cr R_{\rs{XY}} & R_{\rs{YY}}} 
=
\tensor{P}_{M\rightarrow R} \; \tensor{D} \; \tensor{P}_{S\rightarrow M}
\;\;\; \tensor{S} \ \ \  ,
\label{SVD}
\end{equation}
and
\begin{equation}
R_{j,m}(\omega) =
\sum_{n=1}^{N} \, \,
P_{j}^{(n)}(\mathbf{x}_R) \, \,
D^{(n)}(\omega) \, \,
P_{m}^{(n)}(\mathbf{x}_S) \ \ \ .
\label{SVDray}
\end{equation}

Note that the macro for the \verb#\tensor# command has been changed to
force tensors to be bold uppercase, in compliance with current SEG
submission standards. This is so that documents typeset to the old
standards will print out according to the new ones: e.g., tensor
$\tensor{t}$ (note converted to uppercase).

\subsection*{Figures}
\renewcommand{\figdir}{Fig} % figure directory

Figure~\ref{fig:waves} shows what it is about.

\plot{waves}{width=\textwidth}
{This figure is specified in the document by \texttt{
    $\backslash$plot\{waves\}\{width=$\backslash$textwidth\}\{This caption.\}}.
}

\subsubsection{Multiplot} 

Sometimes it is convenient to put two or more figures from different
files in an array (see Figure~\ref{fig:exph,exgr}). Individual plots
are Figures~\ref{fig:exph} and~\ref{fig:exgr}.

\multiplot{2}{exph,exgr}{width=0.4\textwidth}
{This figure is specified in the document by \newline \texttt{
    $\backslash$multiplot\{2\}\{exph,exgr\}\{width=0.4$\backslash$textwidth\}\{This caption.\}}.
}

The first argument of the \texttt{multiplot} command specifies the
number of plots per row.

\subsection{Tables}

The discussion is summarized in Table~\ref{tbl:example}.

\tabl{example}{This table is specified in the document by \texttt{
    $\backslash$tabl\{example\}\{This caption.\}\{\ldots\}}.
}{
  \begin{center}
    \begin{tabular}{|c|c|c|}
      \hline
      \multicolumn{3}{|c|}{Table Example} \\
      \hline
      migration\rule[-2ex]{0ex}{5ex} & 
      $\omega \rightarrow k_z$ & 
      $k_y^2+k-z^2\cos^2 \psi=4\omega^2/v^2$ \\
      \hline
      \parbox{1in}{zero-offset\\diffraction}\rule[-4ex]{0ex}{8ex} &
      $k_z\rightarrow\omega_0$ &
      $k_y^2+k_z^2=4\omega_0^2/v^2$ \\
      \hline
      DMO+NMO\rule[-2ex]{0in}{5ex} & $\omega\rightarrow\omega_0$ & 
      $\frac{1}{4}
      v^2k_y^2\sin^2\psi+\omega_0^2\cos^2\psi=\omega^2$ \\
      \hline
      radial DMO\rule[-2ex]{0in}{5ex} & $\omega\rightarrow\omega_s$ &
      $\frac{1}{4}v^2k_y^2\sin^2\psi+\omega_s^2=\omega^2$\\
      \hline
      radial NMO\rule[-2ex]{0in}{5ex} & $\omega_s\rightarrow\omega_0$ &
      $\omega_0\cos\psi=\omega_s$\\
      \hline
    \end{tabular}
  \end{center}
}
\section{ACKNOWLEDGMENTS}

I wish to thank Ivan P\v{s}en\v{c}\'{\i}k and Fr\'ed\'eric Billette
for having names with non-English letters in them.  I wish to thank
\cite{Cerveny} for providing an example of how to make a bib file that
includes an author whose name begins with a non-English character and
\cite{forgues96} for providing both an example of referencing a Ph.D.
thesis and yet more non-English characters.
\append{Appendix example}
\label{example}

According to the new SEG standard, appendices come before references.

\begin{equation}
\frac{\partial U}{\partial z} = 
\left\{
  \sqrt{\frac{1}{v^2} - \left[\frac{\partial t}{\partial g}\right]^2} +
  \sqrt{\frac{1}{v^2} - \left[\frac{\partial t}{\partial s}\right]^2}
\right\}
\frac{\partial U}{\partial t}
\label{eqn:partial}
\end{equation}
It is important to get equation~\ref{eqn:partial} right. See also
Appendix~\ref{equations}.

\append[equations]{Another appendix}

\begin{equation}
\frac{\partial U}{\partial z} = 
\left\{
  \sqrt{\frac{1}{v^2} - \left[\frac{\partial t}{\partial g}\right]^2} +
  \sqrt{\frac{1}{v^2} - \left[\frac{\partial t}{\partial s}\right]^2}
\right\}
\frac{\partial U}{\partial t}
\label{eqn:partial2}
\end{equation}
Too lazy to type a different equation but note the numeration.

The error comparison is provided in Figure~\ref{fig:errgrp}.

\sideplot{errgrp}{width=0.8\textwidth}
{This figure is specified in the document by \texttt{
    $\backslash$sideplot\{errgrp\}\{width=0.8$\backslash$text\-width\}\{This caption.\}}.
}

\newpage

\bibliographystyle{seg}  % style file is seg.bst
\bibliography{example}

\end{document}
